\subsectionanum{Plan de Acción}
{

\newpar{
EL plan de acción ya se ha desarrollado por bastantes semanas. En la reunión de la semana 6 nos pusimos de acuerdo en ideas generales para organizar mejor este plan.
}

\newpar{
Se optó por dividir el plan de acción en dos partes: el de la creación de un conjunto de listas de supuestos supuestos, y después crear un conjunto de subproblemas encadenados entre sí. A este conjunto de subproblemas se le llamó \textit{Marco de Trabajo}\footnote{O \textit{FrameWork}, como se describe en el libro de donde se obtuvo la idea.}.
}

\newparnl{
A rasgos generales, la idea de un Marco de Trabajo es una especie de \quotes{Caja Negra}, en donde se le puede entregar datos y supuestos, y te retorna la solución del problema, condicionado a las otras dos variables de entrada.
}

}

\subsectionanum{Asignación de Autobuses}
{

\newpar{
Se procedió a pensar en esta problemática, avanzando de lo que se tenía en la semana anterior, y se empezó a redactar las ideas en la bitácora. Se realizaron diversos esquemas y figuras para que las ideas que se intentan transmitir queden bien planteadas. 
}

}