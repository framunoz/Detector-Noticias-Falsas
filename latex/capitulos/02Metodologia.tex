%=======================Plan de Acción===============================
\subsectionanum{Plan de Acción}
{

\newpar{
Para la elaboración del plan de acción, se pensará y discutirá en como se debe de proceder para solucionar el problema planteado por el cliente.
}

\newparnl{
Posteriormente, se elaborarán diagramas que ejemplifiquen mejor las ideas de cómo realizar el plan de acción, y se redactarán las ideas aclarando qué es el problema, cuál es su importancia, para qué servirán las variables de entrada y cuáles serán las variables de salida, indicando también su formato de entrega.
}

}
%=======================Asignación de Autobuses===============================
\subsectionanum{Asignación de Autobuses}
{

\newpar{
Para el problema de la asignación de autobuses, se reflexionará en la naturaleza del problema, se escribirán todas las ideas que se puedan presentar, y se refinarán en el tiempo.
}

\newpar{
Posteriormente, se intentará abstraer de la idea de la asignación de autobuses, para encontrar un modelo que puede ser aplicado en cualquier otro contexto, no solo en un problema de autobuses. Se anotarán las variables de forma que sea lo más entendible y lo más simple posible, como también, de forma que siga manteniendo el espíritu de su antecesor.
}

\newparnl{
Finalmente, se redactará el problema en la bitácora, y se crearán ilustraciones de forma que ejemplifique las ideas que se intentan transmitir.
}

}