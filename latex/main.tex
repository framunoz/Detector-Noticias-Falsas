% Template:     Template Reporte LaTeX
% Documento:    Archivo principal
% Versión:      1.2.3 (03/01/2020)
% Codificación: UTF-8
%
% Autor: Pablo Pizarro R.
%        Facultad de Ciencias Físicas y Matemáticas
%        Universidad de Chile
%        pablo@ppizarror.com
%
% Sitio web:    [https://latex.ppizarror.com/reporte]
% Licencia MIT: [https://opensource.org/licenses/MIT]

% CREACIÓN DEL DOCUMENTO
\documentclass[letterpaper,11pt]{article} % Articulo tamaño carta, 11pt

% INFORMACIÓN DEL DOCUMENTO
\def\titulodelreporte {Marco de Trabajo y Asignación de Autobuses}
\def\temaatratar {Semana 6-7}
\def\fechadelreporte {\today}

\def\autordeldocumento {Francisco Muñoz G.}
\def\nombredelcurso {}
\def\codigodelcurso {Francisco Muñoz G.}

\def\nombreuniversidad {Proyecto Oferta}
\def\nombrefacultad {}
\def\departamentouniversidad {SARCAN Science}
\def\imagendepartamento {departamentos/fcfm}
\def\localizacionuniversidad {Santiago, Chile}

% CONFIGURACIONES
\input{lib/config}

% IMPORTACIÓN DE LIBRERÍAS
\input{lib/env/imports}

% IMPORTACIÓN DE FUNCIONES Y ENTORNOS
\input{lib/cmd/all}

% IMPORTACIÓN DE ESTILOS
\input{lib/style/all}

% CONFIGURACIÓN INICIAL DEL DOCUMENTO
\input{lib/cfg/init}

% INICIO DE LAS PÁGINAS
\begin{document}
	
% CONFIGURACIÓN DE PÁGINA Y ENCABEZADOS
\input{lib/cfg/page}

% CONFIGURACIONES FINALES
\input{lib/cfg/final}

% ======================= INICIO DEL DOCUMENTO =======================

% Título y nombre del autor
\inserttitle

% Resumen o Abstract
\begin{abstract}
	\input{capitulos/00Resumen.tex}
\end{abstract}
\hspace{1.7cm}
% Ejemplo, se puede borrar
%\input{lib/etc/example}
\begin{multicols}{2}[][1]
    \section{Objetivos}
    \input{capitulos/01Objetivos.tex}
    \section{Metodología}
    %=======================Plan de Acción===============================
\subsectionanum{Plan de Acción}
{

\newpar{
Para la elaboración del plan de acción, se pensará y discutirá en como se debe de proceder para solucionar el problema planteado por el cliente.
}

\newparnl{
Posteriormente, se elaborarán diagramas que ejemplifiquen mejor las ideas de cómo realizar el plan de acción, y se redactarán las ideas aclarando qué es el problema, cuál es su importancia, para qué servirán las variables de entrada y cuáles serán las variables de salida, indicando también su formato de entrega.
}

}
%=======================Asignación de Autobuses===============================
\subsectionanum{Asignación de Autobuses}
{

\newpar{
Para el problema de la asignación de autobuses, se reflexionará en la naturaleza del problema, se escribirán todas las ideas que se puedan presentar, y se refinarán en el tiempo.
}

\newpar{
Posteriormente, se intentará abstraer de la idea de la asignación de autobuses, para encontrar un modelo que puede ser aplicado en cualquier otro contexto, no solo en un problema de autobuses. Se anotarán las variables de forma que sea lo más entendible y lo más simple posible, como también, de forma que siga manteniendo el espíritu de su antecesor.
}

\newparnl{
Finalmente, se redactará el problema en la bitácora, y se crearán ilustraciones de forma que ejemplifique las ideas que se intentan transmitir.
}

}
    \section{Desarrollo}
    \subsectionanum{Plan de Acción}
{

\newpar{
EL plan de acción ya se ha desarrollado por bastantes semanas. En la reunión de la semana 6 nos pusimos de acuerdo en ideas generales para organizar mejor este plan.
}

\newpar{
Se optó por dividir el plan de acción en dos partes: el de la creación de un conjunto de listas de supuestos supuestos, y después crear un conjunto de subproblemas encadenados entre sí. A este conjunto de subproblemas se le llamó \textit{Marco de Trabajo}\footnote{O \textit{FrameWork}, como se describe en el libro de donde se obtuvo la idea.}.
}

\newparnl{
A rasgos generales, la idea de un Marco de Trabajo es una especie de \quotes{Caja Negra}, en donde se le puede entregar datos y supuestos, y te retorna la solución del problema, condicionado a las otras dos variables de entrada.
}

}

\subsectionanum{Asignación de Autobuses}
{

\newpar{
Se procedió a pensar en esta problemática, avanzando de lo que se tenía en la semana anterior, y se empezó a redactar las ideas en la bitácora. Se realizaron diversos esquemas y figuras para que las ideas que se intentan transmitir queden bien planteadas. 
}

}
    \section{Resultados}
    \newpar{
Los resultados de ambas problemáticas se pueden observar directamente en la bitácora. Sin embargo, dadas las distintas modificaciones que se han hecho a esta, se puede ver un poco distorsionado los resultados.
}

\newpar{
Con respecto al plan de acción, se puede observar los resultados en la parte \textbf{II Métodos de resolución}. Específicamente, en los capítulos \textbf{4. Plan de Acción}, \textbf{5. Marco de Trabajo} y \textbf{6. Supuestos}.
}

\newparnl{
Por otro lado, se puede observar el progreso del problema de asignación de autobuses en el capítulo \textbf{12. Asignación de Autobuses}.
}
    \section{Objetivos Futuros}
    \newpar{
En esta parte se presentará posibles mejoras que se pueden tener a consideración para un futuro.
}

\newpar{
Con respecto al plan de acción, como la lista de supuestos es relativamente nueva, aún quedan supuestos que se podrían llegar a agregar o quitar. Eso dependerá de las condiciones en que se realice el debate de ciertos supuestos que hagan más complicados o más simple el problema enfrentado.
}

\newpar{
También se puede revisar la redacción y ortografía de las mismas, y analizar si falta más información.
}

\newpar{
Con respecto al problema de asignación de autobuses, aún falta modificar el problema de transporte generalizado al subproblema en que nos vemos enfrentado.
}

\newpar{
También hace falta en considerar los supuestos necesarios para elaborar los distintos modelos, y es necesario añadir varios teoremas o conjeturas, de forma que simplifique en trabajo de crear un modelo para cierto supuesto. Y por último, hace falta revisar la redacción con que se ha elaborado el documento.
}


\end{multicols}

% FIN DEL DOCUMENTO
\end{document}